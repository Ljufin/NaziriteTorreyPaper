\documentclass[12pt,a4paper]{article}
\usepackage[utf8]{inputenc}
\usepackage{amsmath}
\usepackage{amsfonts}
\usepackage{amssymb}

\author{Will}
\begin{document}
\paragraph{Will Gertsch Sophomore Spring Paper Proposal}
\paragraph{Rough Thesis:} The Nazerite oath represents a greater spiritual reality and the keepers of the oath enjoy a special physical and spiritual relationship with God.

\paragraph{Texts used} Numbers chapter 6, Romans, \textit{On the Apostolic Preaching}, \textit{The Sacred Supper of Christ, and What It Brings to Us} from \textit{Institutes of the Christian Religion}, \textit{Lectures on Galatians}.

\paragraph{Strategy} The paper will outline several different topics before arguing the thesis. Therefore, there will be several "chapters" to the overall paper.

\paragraph{Introduction} Introduce the topic and the texts I will be using. State the thesis.

\paragraph{The Oath} Introduce the Numbers passage that I will be using to define what the oath involves and also some other examples from Sampson's life. This part will also raise some questions that will need to be answered in the following sections.  Overall, I want to keep OT based interpretation limited to this section so that it will not take over the rest of the paper.

\paragraph{The Relationship between Physical and Spiritual}  The goal of this section is to show that there is connection between physical things ordained by God and spiritual "realities." Start with an explanation of how the physical is intertwined with the spiritual using \textit{On the Apostolic Preaching}. Then use Romans and Luther to show the relationship between inward righteousness and outward actions(in line with the Law)

\paragraph{Applications to the Oath} This part will take what was learned in the previous part and then apply it to the Nazerite oath in order to understand the spiritual realities of its physical manifestation. Strong use of the method from Calvin's \textit{The Sacred Supper of Christ, and What It Brings to Us} from \textit{Institutes of the Christian Religion.} The earlier work with Luther will also be on the table.

\paragraph{Closing applications} At this point, there should be a clearer picture of what is means to be "separated to God" through the Nazerite vows. Summarize what was argued with a focus on "relatable" parts to a post-Law, Christian society.


\end{document}