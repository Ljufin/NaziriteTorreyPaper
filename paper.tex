\documentclass[12pt]{turabian-researchpaper}
\usepackage[utf8]{inputenc}
\usepackage{amsmath}
\usepackage{amsfonts}
\usepackage{amssymb}
 \usepackage{setspace} 
 \usepackage{alltt}
\author{Will}
\usepackage[T1]{fontenc}
\usepackage{times}
\begin{document}
\thispagestyle{empty}
 \begin{center}
 BIOLA UNIVERSITY\\
 \vspace{4cm}
 ALL THE DAYS OF HIS RIGHTEOUSNESS:\\
 \vspace{0.1cm}
 THE SANCTITY OF THE NAZIRITE VOWS\\
 \vspace{3cm}
 TORREY HONORS INSTITUTE\\
 \vspace{0.1cm}
 DR. FRED SANDERS\\
 \vspace{0.1cm}
 JOHNSON HOUSE\\
 \vspace{0.1cm}
 SOPHOMORE SPRING\\
 \vspace{2cm}
 12 May 2017\\
 \vspace{3cm}
 BY,\\
 \vspace{0.1cm}
 WILLIAM W. GERTSCH
 \end{center}
 \pagebreak
 
 \thispagestyle{plain}
 \setcounter{page}{1}
 \begin{center}
 All the Days of His Righteousness:\\
 The Sanctity of the Nazirite Vows
 \end{center}
 

The Old Testament contains many curiosities that fall outside of a modern reader's understanding and context.
One such curiosity is the Nazirite oath described in Numbers chapter 6. 
These vows most notably form the basis for Samson's life in the book of Judges. 
Similar vows appear in other passages of scripture. 
Therefore, knowledge of what the oath involves will lead to a better understanding of these scenes in scripture. 
This also shows the importance of knowing the particular significance of each vow and the overall significance of the oath. 
These physical components of the oath point to a greater spiritual truth. 
The Narizite oath is an old testament sacrament in that it produces holiness in the keeper of the vows.
The higher significance, and therefore the sacramental nature, of the oath is the result of pointing toward the passive righteousness of Christ. 
The individual vows and promises of the oath within the context of the Law evidence this association.

\begin{center}
\textbf{The Oath}
\end{center}
\par
Before understanding what the oath entails in a spiritual sense, the physical objects of the oath must be well understood. 
The vows are an old sacrament in that they serve a similar purpose to the elements of the holy communion, of which Calvin says, "Since, however, this mystery of Christ's secret union with the devout is by nature incomprehensible, he shows its figure and image in visible signs best adapted to our small capacity."\footnote{Jean Calvin, \textit{Calvin: Institutes of the Christian Religion: Volume 2}, trans. Ford Lewis. Battles, ed. John T. McNeill (Louisville, KY: Westminster John Knox Press, 2011), 1361.} 
Therefore, knowledge of the physical qualities of the vows will lead to a greater understanding of the spiritual figure of the Nazirite oath. 

\par
The Nazirite oath exists within a larger section in Numbers chapter 2 though chapter 9 that details the arrangement of Israel's camp in the wilderness.
The Nazirite oath itself is described in chapter 6.
Verses 1-8 contain the vows that the Nazitite must keep and verses 9-21 describe ceremonies related to the oath.
The specifications for the vows begin in the first two verses, which read, "And the LORD spoke to Moses, saying 'Speak to the people of Israel and say to them, When a man or woman makes a special vow, the vow of a Nazirite, to separate himself to the LORD...'"  \footnote{Num. 6:1-2.}
These two verses establish the relationship between those who keep the oath and the LORD.
The concept of separation is important to understanding the oath as a whole.
\par
The next few verses detail the first two vows of the oath.
The first vow states the Nazirites's required abstinence from wine.
"He shall separate himself from wine and strong drink. He shall drink no vinegar made from wine and strong drink and shall not drink any juice of grapes or eat grapes, fresh or dried. All the days of his separation he shall eat nothing that is produced by the grapevine, not even the seeds or the skins. \footnote{Num. 6:3-4.}
The second vow involves the Nazirite's hair. 
All the days of his vow of separation, no razor shall touch his head. Until the time is completed for which he separates himself to the LORD, he shall be holy. He shall let the locks of hair of his head grow long." \footnote{Num. 6:5.} 
These two vows serve a similar purpose in terms of the Nazirite's separation to the Lord.
The abstinence from wine shows the Nazirite's removal from the sustaining things of the earth and a new reliance on the abstinence found in the Lord.
Similarly, the Nazirite's hair demonstrates the individual's relationship with the Lord.
Therefore, these two vows primarily signal the Nazirite's separation to the Lord.
\par
The third and final vow concerns the Nazirite's contact with the dead.
"All the days that he separates himself to the LORD he shall not go near a dead body. Not even for his father or for his mother, for brother or sister, if they die, shall he make himself unclean, because his separation to God is on his head. All the days of his separation he is holy to the LORD." \footnote{Num. 6:6-8}
To see the true significance of this vow, it is necessary to observe the preceding chapter in Numbers. 
In chapter 5, God expels certain groups from the camp for their uncleanness. 
One group to be removed is "everyone who is unclean through contact with the dead." \footnote{Num. 5:2.}
Therefore, contact with the dead is already considered unclean, even outside of the Nazirite vow. 
The Lord's purpose for this removal hints at the significance to the Nazirite oath.
The Lord commands their removal so that "they may not defile their camp, in the midst of which I dwell." \footnote{Num. 5:4.} 
Therefore, just as the unclean are removed from the camp, so must the Nazirite remove his uncleanness from the presence of God, for the Nazirite is separated to the Lord.
Numbers 6:9 states that contact with the dead "defiles his consecrated head." \footnote{Num. 6:9.} 
Hence, avoiding the dead is essential to the Narizite's holiness. 

 \pagebreak

\begin{center}
\textbf{Spirituality and Righteousness}
\end{center}

\par
Now that the physical vows have been examined, the Mosaic Law must be shown to have spiritual significance. 
This connection establishes the spiritual meaning of the oath since the vows are a subset of the overall Law. 
The spiritual implications of Law also establish a relationship with passive righteousness.

\par
Beginning with the Mosaic Law, this law has a higher significance because of its direction revelation to Israel from God.
Paul states that the Law serves to reveal man's unrighteousness and to compare it against God's righteousness.
He says, "Yet if it had not been for the law, I would not have known sin," \footnote{Rom. 7:7.} and "...our unrighteous serves to show the righteousness of God." \footnote{Rom. 3:5.}
Therefore, the Law serves to remind man of his place under God. 
This demonstrates the Law's relationship to the spiritual things of God and therefore also shows how the Nazirite oath fits into this purpose.

\par
The role of the Law in conviction and worship explains the general importance of the Law, but this broad application also helps to understand how the specific commands of the Law fit into the spiritual. 
This requires an additional step of reasoning. 
For example, it is trivial to say, "sacrifice is spiritual as well as physical because it reminds man of his place," but it says nothing of the deed itself, and does not reveal anything specific about the action. 
The required step is to distinguish how the particular action is distinct from other actions that accomplish the same purpose. 
For example, the sacrifice of animals for the remission of sin serves the purpose of exalting God not man, but the way in which the action serves the end is distinct from other acts of the Law.
Sacrifice and cleanness might serve the same ultimate purpose, but they are not at all the same thing. 
Knowing how the action serves the end reveals more about the action than otherwise would be apparent from a purely physical or spiritual point of view.
To summarize, understanding the Law's purpose of revealing the relationship between man and God also helps to understand the spiritual significance of a specific physical action of the Law. 

\par
With the Law's connection to the spiritual defined, it remains to discuss the Law's role in righteousness. 
Luther divides righteousness into two categories when he says, "But this most excellent righteousness, the righteous of faith ... is neither political nor ceremonial nor legal nor work-righteousness but is quite the opposite; it is a merely passive righteousness, while all the others, listed above, are active." \footnote{Luther, Martin, \textit{Martin Luther's Basic Theological Writings}, edited by Timothy F. Lull and William R. Russell, (Minneapolis: Fortress Press, 2012), 91.} 
Luther refers to the righteousness of faith as passive righteousness. 
Everything else, he classifies as active righteousness. 
Luther values passive righteousness over active righteousness and is correct in making this evaluation since Paul writes, "we have been justified by faith." \footnote{Rom. 5:1.} 
But this does not mean that the Law is evil or is meaningless because Paul also says, "... the law is holy, and the commandment is holy and righteous and good." \footnote{Rom. 7:12.}  
Luther also makes this distinction when he says "The righteousness of the law is earthly and deals with earthly things; by it we perform good works." \footnote{Luther, 89.}
Further, Luther also says that the law of the flesh, or active righteousness, rules over man, when man is still of the flesh, even though justification is through passive righteousness.
However, in the context of Law's ultimate purpose, the active righteousness of the Law specifically looks forward to fulfillment in the passive righteousness of Christ.
Proof of this is that Abraham's faith is "counted to him as righteousness." \footnote{Rom. 4:22.}
Since God's relationship with Israel began with Abraham's faith, the Law has a special relationship with passive righteousness.
Namely, the Law discourages self-righteousness and promotes the passive righteousness of Abraham by revealing the sin of man and the holiness of God.
Therefore, the Law ultimately serves to direct humanity to Christ's righteousness.

\begin{center}
\textbf{Understanding the Oath}
\end{center}
\par

The statements made in the section on the physical vows and the overall purpose of the oath made in the section after will aid in understanding the  particulars of the Nazirite oath. 
In addition, Calvin provides machinery for understanding how the physical components of the oath relate to the overall purpose of the law and to righteousness.

\par
In his work on the sacrament of communion in \textit{The Institutes}, Calvin outlines a method of understanding how the elements of communion signify Christ's relationship with the Christian. 
He does this because, "...the sacred mystery of the Supper consists in two things: physical signs, which, thrust before our eyes, represent to us, according to our feeble capacity, things invisible; and spiritual truth, which is at the same time represented and displayed through the symbols themselves." \footnote{Calvin, \textit{Institutes of the Christian Religion: Volume 2}, 1371.} 
Similarly, the Nazirite oath's significance of higher things through the Law and similarity to the sacraments make Calvin's approach useful in this instance.
Calvin's method works as follows, "... I usually set down three things: the signification, the matter that depends upon it, and the power or effect that follows from both" \footnote{Calvin, \textit{Institutes of the Christian Religion: Volume 2}, 1371.} 
Therefore, the Nazirite oath should also be set down in three things.

\par
The first thing to be set down is the signification.
In Calvin's original use of his method, the signification of communion belonged in the promises contained in the signs of communion. 
Similarly, the signs of the Nazirite, the abstinence from wine and his hair, represent the promise contained in the oath of the Nazirite's separation to the Lord.
Therefore, the Nazirite vows signify this separation.


\par
Calvin's next category is the "matter that depends on it," which, in relation to Communion, is "Christ with His death and resurrection." \footnote{Ibid, 1372.} 
In this instance, understanding what the Nazirite participates in will clarify what Calvin means by "matter" or by "substance." 
In Calvin's original argument, the believer participates in Christ.
In the Nazirite oath, the Nazirite participates in cleanness.
Calvin's example also shows the importance manifest in the substance of a sacrament.
The redemption of man, named as an effect of Communion by Calvin, is only possible through Christ's defeat of sin.
Without Christ, the signification of Communion does not imply any higher meaning. 
In the same way, the separation of the oath cannot achieve anything without cleanness, for God commanded that the unclean be put out as to not defile the camp, in which he dwells. 
Similarly, God commands the removal of the defiled from His presence that the Nazirite enters into by his separation to the Lord.
Therefore, the Nazirite's participation in cleanness enables the signification of the oath and produces the effects of the sacrament.

\par
The final category is the "power or effect that follows from both."
Notably, the physical action of the oath does not produce the effect. 
This is apparent since neither the signification nor the matter are  ultimately physical. 
Calvin's particular case demonstrates this, for the effects of Communion are "redemption, righteousness, sanctification, and eternal life" \footnote{Calvin, \textit{Institutes of the Christian Religion: Volume 2}, 1371.} 
Physical action, or active righteousness, does not cause these, instead, they live in the domain of passive righteousness.
Since the signification and matter of the Nazirite oath are spiritual, the effects of the oath are also spiritual. 
This does not restrict any physical benefits, but these blessings are not the primary effect of the oath just as the physical nourishment of the Supper is not its primary effect. 
The primary effect of the Narizite's separation and cleanness before God is that the Nazirite becomes holy. 
The oath's description, which says, "Until the time is completed for which separates himself to the LORD, he shall be holy" \footnote{Num. 6:5.} and "All the days of his separation he is holy to the LORD," \footnote{Num. 6:8.} evidences this effect. 
Therefore, the Nazirite oath produces holiness before God.


\par
With these things in mind, the Nazirite oath reflects the spiritual relationship of the Christian with God. 
More specifically, the oath foreshadows the passive righteous that came later through Christ.
It does this within the context of active righteousness.
The individual vows of the oath signify things of passive righteousness that belong to those who share in the faith of Abraham.
Firstly, the Nazirite's physical distinction from the rest of Israel reflects how Christians are separated from the old and joined to the new.
This similarity continues when comparing the outward perception of the Nazirite and the Christian by others.
A non-Nazirite could easily keep the vows outwardly but lack the initial vow of separation to the Lord.
This fits into Luther's model that passive righteous and true justification in Christ is something not outwardly apparent through active righteousness. 
The most significant comparison of the activeness of the Nazirite oath and passive righteousness is found, not in any of the three primary vows, but in cleansing ceremony after the Nazirite has been defiled by contact with a dead body.
This ceremony, described in Numbers 6:9-12, essentially resets the duration of the Nazirite oath. 
The Nazirite is not excluded from the vow because of his contact with a dead body, but is instead renewed in the separation to the Lord.
This points to the redemption found in passive righteousness in that sin does not nullify justification and the believer is continually renewed in Christ. 

\par
Therefore, the comparison between the physical actions of the Nazirite oath and the things of passive righteousness is a fitting one. 
The Nazirite oath, an old sacrament, foreshadows the new covenant in Christ.  
In terms of active and passive righteousness, the active righteousness of the Narizite oath looks towards the passive righteousness in justification by faith, and the objects of the oath symbolize the things that the Christian enjoys in Christ. 
However, since the oath serves to point to its fulfillment, a Christian observing the vows may lead to inconsistencies.
For example, a Nazirite Christian would avoid wine in observance of the vow but also drink wine in keeping of the Holy Supper.
Therefore, the fulfillment in purpose of the Nazirite oath by Christ renders the oath absurd and pointless in comparison to its real completion.
The Nazirite oath now only serves to reveal, through the use of its physical commandments, the greater holiness of Christ.

\pagebreak
\singlespace
\begin{center}
Works Cited
\end{center}
\begin{alltt}
\normalfont
Jean Calvin, \textit{Calvin: Institutes of the Christian Religion: Volume 2}, translated
             by Ford Lewis Battles, edited by John T. McNeill, Louisville, KY: Westminster John Knox Press, 2011
\end{alltt}





\begin{alltt}
\normalfont
Luther, Martin, \textit{Martin Luther's Basic Theological Writings}, edited by Timothy F. Lull 
             and William R. Russell, Minneapolis: Fortress Press, 2012
\end{alltt} 


\begin{alltt}
\normalfont
ESV Bible, Thinline Compact Edition. Wheaton: Crossway, 2001
\end{alltt}


\end{document}